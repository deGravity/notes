\documentclass{article}
\usepackage{macros}
\usepackage{wasysym}

\begin{document}
\section{Well Posed Problems}
Heat Equation:\\

Dirchlet Problem
\begin{align*}
\text{DE:}&~ u_t = Du_{xx} , 0< x < l, t > 0\\
\text{IC:}&~ u(x,0) = f(x) \tag{D}\\
\text{BC:}&~ u(0,t) = g(x)
\end{align*}

Neumann Problem
\begin{align*}
\text{DE:}&~ u_t = Du_{xx}, 0 < x < l, t > 0\\
\text{IC:}&~ u(x,0) = f(x) \tag{N}\\
\text{BC:}&~ u_x(0,t) = a(t), u_x(l,t) = b(t)
\end{align*}

Wave Equation:
\begin{align*}
\text{DE:}&~ u_tt = c^2 u_xx, 0 < t, 0 < x < l\\
\text{BC:}&~ u(0,t) = 0, u(l,t) = 0 \tag{W}\\
\text{IC:}&~ u(x,0) = f(x), u_t(x,0) = y(x)
\end{align*}

Why these three problem? They are examples of well posed problems.

A well posed problem has 3 characteristics
\begin{enumerate}
	\item[1)] Existence: A solution exists to the problem.
	\item[2)] Uniqueness: The solution is unique.
	\item[3)] Stability: If a small change is made in the initial condition or boundary condition, the solution changes by only a small amount.
\end{enumerate}

We need to also talk about regularity. Solutions ``live'' in a function space. For example, for the heat equation, it is natural to talk about $u(x,t) \in C_x^2[0,l]$ - that is $u(x,t),$ $u_x(x,t),$ and the second derivative are continuous, and $u(x,t) \in C_t^1[0,\infty)$ - that is u(x,t) and $u_t(x,t)$ are continuous.

\subsection{Existence}
Existence usually (for this course) is demonstrated by an explicit solution. For example,

$$u(x,t) = \sum_{n=1}^{N} e^{-D(n^2 \pi^2 / L^2)t}\sin(\frac{n \pi x}{L})$$
is a solution to (D) (Dirchlet problem) with 
$a(t) = b(t) = 0$ and $f(x) = \sum_{n=1}^{N} a_n \sin(\frac{n \pi x}{L}).$

There is another method, called fixed point analysis, in which you bound the solution by a strictly decreasing sequence of sets in function space that converges to a point.

To show that a solution does not exist, one must usually derive a contradiction.

\subsection{Uniqueness}
To show uniqueness for linear problems, one almost always starts the same way. Proof by contradiction:
Suppose $u_1(x,t)$ and $u_2(x,t)$ are two solutions of the Dirchlet problem
\\~\\
Let $v = u_1(x,t) - u_2(x,t)$. Note
$$v_t - Dv_xx = (u_1)_t s D(u_1)_xx) - ((u_2)_t - D(u_2)_xx) = 0;$$
So the DE for v is
\begin{align*}
\text{DE:}&~ v_t = Dv_xx \quad 0 < x < l\\
\text{IC:}&~ v(x,0) = 0 \quad 0 < x < l \text{\footnotemark} \tag{\smiley} \\
\text{BC:}&~ v(0,t) = 0, v(l,t) = 0, \quad t \geq 0.
\end{align*}
\footnotetext{\text{For example,} $v(x,0) = u_1(x,0) - u_2(x,0) = f(x) - f(x) = 0$.}
I need to show $v= 0$ is the only solution.

\subsubsection{Energy Methods}

Let $E[v] = \int_0^l \frac{v^2}{2} ~dx$. Note $E[v]$ is a function of time only.
What is $\frac{dE}{dt}$, assuming v satisfies (\smiley)?

\begin{align*}
\frac{dE[v]}{dt} &= \frac{d}{dt} \int_0^l \frac{v^2}{2} dx\\
&=  \int_0^l \frac{d}{dt} \frac{v^2}{2} dx\\
&= \int_0^l v v_t dx\\
&= \int_0^l v(Dv_xx)~dx.
\end{align*}
Note that since the boundaries of the integral do not depend on time, the derivative with respect to time can move inside. Integrate by parts

\begin{align*}
p&=v, \qquad dp = v_x dx\\
dq&= v_xx dx, \qquad q = v_x
\end{align*}

$$\frac{dE}{dt} = D \Big[ pq [_0^l - \int q dp ]$$

So

$$\frac{dE}{dt} = D[vv_x [_0^l - \int_0^l v_x v_x dx ].$$

Note that v $v_x$ vanishes by the (\smiley) boundary conditions, and

$$\frac{dE}{dt} = - D \int_0^l (v_x)^2 dx.$$

Note, $\frac{dE}{dt} \leq 0$ which implies that $E$ is non-increasing. Also,
$$E = \int_0^l \frac{v^2}{2} ~dx \geq 0,$$
and
$$E[v(0)] = \int_0^l \frac{0^2}{2} ~dx = 0,$$
so $E$ is initially $0$, always non-negative, and non-increasing. Thus $E = 0$ for all $t > 0$. 
Note that this implicitly used the continuity of u in time.
If $E \geq 0$ and continuous, then $v = 0$  for all $t > 0$; therefore $u_1 = u_2$ and the solution is unique.

\subsection{Stability}
``Can a butterfly flapping its wings in Beiking alter the weather in San Francisco?'' - Paraphrase of Ed Lorenz

If a system is stable, and you make a small change, things don't change much. If a system is not stable, small changes make a huge difference.

An example of instability is the backwards heat equation. Recall that in the heat equation we assume $D$ is positive. Suppose in the Dirchlet problem that $D < 0$. Then heat flows from cold to hot. Note that in our previous derivation of the solution, we did not make any use of the sign of $D$.
\begin{align*}
\text{DE:}& u_t = Du_{xx} \quad 0 < x < l, \quad D < 0 \\
\text{DC:}& u(0,t) = 0, u(l,t) = 0, \quad t > 0 \\
\text{IC:}& u(x,0) = \frac{1}{n} \sin(\frac{n \pi x}{l}) \quad 0 < x <l
\end{align*}
The solution is
$$u(x,t) = \frac{1}{n} \sin(\frac{n \pi x}{l} e^{-D \frac{n^2 \pi^2}{l^2} t}.$$
Note that
$$\max u(x,0) = \frac{1}{n} \qquad 0 < x < l$$
but
$$ \max_{0 < x < l} u(x,t) = \frac{1}{n} e^{-D \frac{n^2 \pi^2 }{l^2} t}.$$
So given any $\delta > 0$, I can choose $n$ such that $\frac{1}{n} < \delta$ and $|u(x,0)| < \delta$.
But at $t = 1$
$$\max_{0 < x < l} u(x,t) = \frac{1}{n} e^{-D\frac{n^2 \pi^2}{l^2}}$$
and as $n \to \infty$, this max tends towards infinity. It turns out that for a generic initial condition, the temperature goes to infinity in a finite amount of time.

In fact, the forward heat equation is stable. As a handwaved argument, let
$$|u_1(x,0) - u_2(x,))| \leq \delta,$$
so in the energy derivation
$$0 \leq \int_0^l \frac{v^2}{2} ~dx \leq \delta^2,$$
so we have convergence in the $L^2$ norm.

\end{document}
