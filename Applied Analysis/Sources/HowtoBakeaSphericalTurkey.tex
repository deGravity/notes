\documentclass[cm]{article}
\usepackage{macros}

\title{How to Cook a Spherical Turkey}

\begin{document}
\maketitle
First imagine a hot copper ball dropped into a heat bath maintained at $0$.
\begin{align*}
\text{DE:}&~ u_t = k \Delta u \qquad x \in \Omega, t > 0 \\
\text{IC:}&~ u(x,0) = T_0 \qquad x \in \Omega \\
\text{BC:}&~ u(x,t) = 0 \qquad x \in \partial \Omega, t > 0
\end{align*}
Where $\Delta = \nabla^2$.
Spherical coordinates would be nice here. So make the following assumption of
spherical symmetry:
$$u = u(r, \phi, \theta, t) = u(r,t).$$
Note that
$$u = u(x) = u(r) \qquad r = |x|.$$
In this case,
$$\Delta u = u_{x_1x_1} + u_{x_2x_2} + \cdots + u_{x_nx_n}$$
\begin{align*}
\frac{\partial u}{\partial x_i} &= u'(r) \frac{\partial r}{\partial x_i} \\
        &= u'(r) \frac{x_i}{r} \\
\frac{\partial^2 u}{\partial x_i^2} &= \frac{u'(r)}{r} + x_i \frac{r u''(r)
    \frac{x_i}{r} - u'(r) \frac{x_i}{r}}{r^2} \\
    &= \frac{u'(r)}{r} + \frac{x_i^2}{r^2} [ ru'' - u']
\end{align*}
So
\begin{align*}
\Delta u &= \nabla^2 = \sum_{i = 1}^{n} u_{x_ix_i} \\
            &= n \frac{u'}{r} + \frac{r^2}{r^3} [ru'' - u'] \\
            &= u'' + \frac{n-1}{r} u'
\end{align*}
which is the {\bf radial Laplacian}.
The PDE now becomes:
\begin{align*}
\text{DE:}&~ u_t = k(u_{rr} + \frac{2}{r} u_r \qquad 0 < r < \pi, t > 0 \\
\text{IC:}&~ u(r,0) = T_0 \qquad x \in \Omega u < r < \pi\\
\text{BC:}&~ u(\pi,t) = 0 \qquad t > 0.
\end{align*}
This is the problem of {\bf spherical cooling}. We can solve this using
separation of variables. Assume
$$u(r,t) = T(t) R(r).$$
Then
\begin{align*}
T'R &= k(TR'' + \frac{2}{r} T R') \\
\frac{T'}{kT} &= \frac{R'' + \frac{2}{r} R'}{R} = - \lambda.
\end{align*}
Our PDEs are thus
\begin{align*}
T' &= - \lambda k T \\
R'' + \frac{2}{r} R' + \lambda R &= 0
\end{align*}
We see the $t$ component has solution:
$$T_\lambda(t) = r_\lambda e^{-k \lambda t},$$
which suggests that $\lambda$ will be positive. This is not a proof, however, so
we look at the second problem:
\begin{align*}
\text{DE:}&~ R'' + \frac{2}{r} R' + \lambda R = 0 \\
\text{BC:}&~ R(\pi) = 0 \qquad ( u(\pi, t) = T(t) R(\pi) = 0 ) \\
\text{HBC:}&~ R(0) < \infty \qquad (R(0) \text{ bounded} ).
\end{align*}
Where (HBC) is a hidden boundary condition. Note the sign of $\lambda$. Given
$$-\Delta u = \lambda u,$$
(e.g $-\Delta u = \lambda u$):
\begin{align*}
(u'' + \frac{n-1}{r} u') &= - \lambda u \\
r^{n-1} u'' + (n-1) r^{n-2} u' + \lambda r^{n-1} u &= 0 \\
\int_0^\pi[r^{n-1} u']'u + \lambda \int_0^pi r^{n-1} u^2 &= 0 \\
r^{n-1}u'u\Big|_0^\pi r^{n-1}(u')^2i~dr + \lambda \int_0^\pi r{n-1} u^2~dr &= 0\\
\lambda &= \frac{\int_0^\pi r^{n1} u(u')^2~dr}{\int_0^\pi r^{n-1} u^2~dr} \\
&\geq 0. 
\end{align*}
Let $\lambda = 0$, then $\int_0^\pi r^{n-1} (u)^2 ~dr = 0$, so $u$ is constant.
Recall we are looking at the Sturm Louiville Problem
$$-(R'' + \frac{2}{r} R') = \lambda R,$$
which can be written
$$-\Delta u = \lambda u.$$ Also, note that
$$ \Delta u + \lambda u = 0$$ is called the Helmholtz Equation. That asside,
a constant $u$ is a problem for our boundary conditions - the function is $0$ on
the boundaries, so $u = 0$ uniformly in this case! Thus we know that $\lambda >
0$. We thus re-write things again:
\begin{align*}
\lambda &= \mu^2 > 0 \\
\text{IC:}&~ R'' + \frac{2}{r} R' + \mu^2 R = 0 \\
\text{HBC:}&~ R(0) \text{ bounded} \\
\text{BC:}&~ R(\pi) = 0
\end{align*}
Then
\begin{align*}
rR'' + 2R' + \mu^2 rR &= 0 \\
Y'' + \mu^2 Y &= 0 \\
Y(0) & = 0 = Y(\pi) \\
Y_{mu}(r) &= A \cos \mu r + B \sin \mu r \\
        &= 0 = \sin \mu \pi \\
        \implies&~ \mu = n \in \N \\
Y_n(r) &= \sin(nr)
\end{align*}
This implies
$$R_n(r) = \frac{\sin(nr)}{r}.$$
Thus the Eigenmodes are
\begin{align*}
u_n(r,t) &= T_n(t) R_n(r) \\
    &= e^{-n^2 k t } \frac{\sin(nr)}{r}.
\end{align*}
Note that this does not have a singularity at the origin because
$$ \lim_{r \to 0} \frac{\sin(nr)}{r} = n.$$ Assume
$$u(r,t) = \sum_{n = 1}^{\infty} A_n e^{-n^2 k t} \frac{\sin(nr)}{r}$$
Let $t = 0$:
$$u(r,0) = T_0 = \sum_{n = 1}^{\infty} A_n \frac{\sin(nr)}{r},$$
so
$$(r T_0) = \sum_{n = 1}^{\infty} A_n \sin(nr),$$
\begin{align*}
A_n &= \frac{2}{\pi} \int_0^\pi T_0 r \sin(nr)~dr = \frac{2 T_0}{\pi}
\int_0^\pi r \sin(nr)~dr \\
&= \frac{2 T_0}{\pi} [ \frac{- r \cos n r}{n}\Big|_0^r + \int_0^\pi \frac{
    \cos(nr)}{r} ~ dr \\
&= \frac{2 T_0}{\pi} \left[ \frac{-\pi (-1)^n}{n} \right] \\
&= \frac{2 T_0}{n} (-1)^{n+1}
\end{align*}
Putting everything together
$$u(r,t) = \sum_{n = 1}^{\infty} \frac{(-1)^{n+1} 2 T_0}{n} e^{-n^2 k t}
\frac{\sin(nr)}{r}.$$
Note, if $\Omega = B(0,\xi)$, then
$$u(r,t) = 2 T_0 \sum_{n=1}^{\infty} (-1)^{n+1} e^{-k (\frac{n \pi}{\xi})^2 t}
\frac{\sin(\frac{n \pi}{\xi} r)}{\frac{n \pi}{\xi} r}.$$
Note, we know
$$\int_0^\pi Y_n(r) Y_m(r)~dr = 0 \qquad m \neq n,$$
so
$$\int_0^\pi R_n(r) R_m(r) r^2~dr = 0 \qquad m \neq n.$$
We say they are orthogonal with respect to ``weight $r^2$.'' \\
Given
\begin{align*}
\text{DE:}&~ u_t = k \Delta u \\
\text{IC:}&~ u(r,0) = u_0(r) \\
\text{BC:}&~ u(R,t) = f(t)
\end{align*}
Let us solve {\bf The Turkey Problem}; putting a cold turkey into a heat bath.
\begin{align*}
\Delta u_s &= 0\\
u_s &= C.
\end{align*}
Solving this, we have,
\begin{align*}
v &= u - u_s \\
v_t &= u_t \\
\Delta v = \Delta u \\
v(R,t) &= C - C = 0 \\
v(r,0) = u_0(r) - u_s(r)
\end{align*}
so the new problem is
\begin{align*}
\text{DE:}&~ v_t &= k \Delta v \\
\text{IC:}&~ v_0(r) \\
\text{BC:}&~ v(R,t) = 0.
\end{align*}
We can now  solve the turkey problem. Suppose we have turkey of roughly
spherical shape that has been defrosted to $75^{\circ}$, and is placed into a
$350^\circ$ oven. Assume $R = 1$, and $k = 0.02$ (this is roughly correct). How
long until the temperature at the center is $150^{\circ}$. (Really it should be
$165^{\circ}$, but we're living on the edge.)
\begin{align*}
\text{DE:}&~ u_t = k \Delta u \qquad 0 < r < 1, t > 0 \\
\text{IC:}&~ u(r,0) = 75 \qquad 0 < r < 1\\
\text{BC:}&~ u(1,t) = 350 \qquad t > 0.
\end{align*}
By our previous work,
$$u = v + w$$
with
$$\begin{cases} \Delta w = 0 \\ w = 350 \end{cases} \implies w = 350$$
$u = v + 350$. Thus
$$\begin{cases} v_t = k \Delta v \\  v(r,0) = -275 \\ v(1,t) = 0 \end{cases}
\implies v(r,t) = -550$$
So
$$u = 350 - 550 \sum \text{(modes)}.$$
Cut to Mathematica worksheet.
\end{document}
