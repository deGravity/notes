\documentclass[cm]{article}
\usepackage{macros}
\usepackage{wasysym}
\usepackage{diagrams}

\renewcommand{\F}{\mathscr F}
\newcommand{\fhat}{\hat{f}}
\renewcommand{\infint}{\int_{-\infty}^{\infty}}
\newcommand{\ghat}{\hat{g}}
\newcommand{\uhat}{\hat{u}}
\newcommand{\Fhat}{\hat{F}}


\begin{document}
\section{Last Time}
Fourier Transform: We had annoying problems with $\xi$ notation, so now
\begin{align*}
\F \{ f(x) \} = \fhat(k) = \int_{-\infty}^{\infty} f(x) e^{-ikx}~dx \\
\F^{-1} \{\fhat(k)\} = f(x) = \frac{1}{2\pi} \infint \fhat(k) e^{ikx}~dx
\end{align*}

\section{Properties of the Fourier Transform}
\enum
    \item Existence: If $f(x)$ is absolutely integrable, that is
    $$ \infint |f(x)|~dx = M < \infty$$
    then $\fhat(k)$ exists.
    \item Uniqueness: Suppose $f(x)$ and $g(x)$ are absolutely integrable and
    $$\fhat(k) = \ghat(k).$$
    Then
    $$\infint |f(x) - g(x)|~dx = 0.$$
    If in addition, $f$ and $g$ are continuous, then $f(x) = g(x)$ almost
    everywhere.
    \item Derivatives: Suppsoe $f(x)$ is absolutely integrable and
    $$\F\{f(x)\} = \fhat(k).$$
    Then
    $$\F\{f'(x)\} = ik \fhat(k).$$
    Note
    \begin{align*}
        \F\{f''(x)\} &= \F\{(f'(x))'\} \\
            &= ik \F\{f'(x)\} \\
            &= (ik)^2 \fhat(k)
    \end{align*}
    \item Some Examples
        \enum
            \item Top Hat:
            $$f(x) = \begin{cases} 1 \quad |x| < L \\ 0 \quad |x| \leq L
            \end{cases}$$
            then
            $$\F\{f(x)\} = \int_{-L}^L 1 e^{-ikx}~dx =  \frac{2 \sin kt}{k}.$$
            \item Heaviside Funtion (Include Graph)
            $$H(x) = \begin{cases} 0 \quad x < 0 \\ \frac12 \quad x = 0 \\ 1 \quad x > 0
            \end{cases}$$
            Think about this as a switch - if you multiply a function by $H(x)$,
                  it switches on at $x = 0$.
            \item $f(x) = H(x)e^{-ax} \quad \Re\{a\} > 0$
            \begin{align*}
                \F\{f(x)\} &= \infint H(x) e^{-ax} e^{-ikx}~dx \\
                    &= \int_0^{\infty} e^{-(a+ik)x}~dx \\
                    &= \frac{1}{a+ik}
            \end{align*}
        \xenum
        \item Linearity:
        $$\F\{af(x) + bg(x)\} = a\fhat((k) + b\ghat(k).$$
        \item Reflection:
            \begin{align*}
            \F\{f(-x)\} &= \infint f(-x) e^{-ikx}~dx \qquad ~z = -x \\
            &= \infint f(z) e^{ikz}(-dz) \qquad dx = - dz\\
            &= \int_{z=-\infty}^{\infty} f(z) e^{-i(-k)z}~dz \\
            &= \fhat(-k)
            \end{align*}
        \item Compute for $\Re\{a\} > 0$
        \begin{align*}
        \F\{e^{-a|x|}\} &= \F\{H(x)e^{-ax} + H(-x)e^{ax}\} \\
        &= \F\{H(x) e^{-ax}\} + \F\{H(-x) e^{ax}\} \\
        &= \frac{1}{a+ik} + \frac{1}{a-ik} \\
        &= \frac{2a}{a^2 + k^2}
        \end{align*}
        \item Shifting:
        \begin{align*}
        \F\{f(x+a)\} &= \infint f(x+a) e^{-ikx}  \quad \qquad ~~z = a+x \\
        &= \infint f(z) e^{-ik(z-a)}~dz \qquad dz = dx \\
        &= e^{ika} \infint f(z) e^{-ikz}~dz \\
        &= e^{ika} \fhat(k)
        \end{align*}
\xenum

\begin{table}
\begin{centering}
\begin{tabular}{|c | c|}
    \hline
    $f(x)$ & $\fhat(k)$ \\
    \hline \hline
    $f'(x)$ & $ik\fhat(k)$ \\ \hline
    $f^{(n)}(x)$ & $(ik)^n\fhat(k)$\\ \hline
    $f(x) = \begin{cases} 1 \quad |x| < L \\ 0 \quad |x| \geq L \end{cases}$ &  $\frac{2\sin
        (kL)}{k}$ \\ \hline
    $H(x) e^{-ax}$ &  $\frac{1}{a+ik}$ \\ \hline
    $f(-x)$ &  $\fhat(-k)$ \\ \hline
    $e^{-a|x|}$ &  $\frac{2a}{a^2 + k^2}$ \\ \hline
    $f(x+a)$ &  $e^{ika} \fhat(k)$\\ \hline
    $\F \{ f(bx) \}$ & $ \frac{1}{b} \fhat \left( \frac{h}{b} \right)$ \\
    \hline
\end{tabular}
\end{centering}
\caption{Table of Fourier Transform Identities Derived in Notes.}
\end{table}

\section{Uses of the Fourier Transform}
Consider the following PDE:
    \begin{align*}
    \text{DE:}&~ u_t + cu_x = 0 \qquad - \infty < x < \infty, t > 0 \\
    \text{IC:}&~ u(x,0) = F(x) \qquad - \infty < x < \infty
    \end{align*}
Solution: Note, if we Fourier Transform in $x$
    \begin{align*}
    \F_x\{u(x,t)\} &= \infint u(x,t) e^{-ikx}~dx \equiv \uhat(k,t) \\
    \F_x\{u_t(x,t)\} &= \infint u_t(x,t) e^{-ikx}~dx = \frac{\partial}
    {\partial t}
    \infint u(x,t) e^{-ikx}~dx \\
    &= \frac{\partial \uhat}{\partial t}(k,t)
    \end{align*}
Also
    $$\F\{u_x(x,t)\} = uk\uhat(k,t).$$
Fourier the Transform of the DE
\begin{equation*}
\F\{u_t + cu_x\} = \uhat_t + ikc \uhat = 0. \tag{\smiley}
\end{equation*}
This is an ODE in $t$. We can solve this:
$$\uhat(k,t) = A(k) e^{-ikct}.$$
To solve for $A(k)$, FT the IC
$$\F\{u(x,0)\} = \uhat(k,0) = \F\{F(x)\} = \Fhat(k).$$
Now
$$\uhat(k,0) \stackrel{\smiley}{=} A(k) =(IC) \Fhat(k)$$
and
$$\uhat(k,t) = \Fhat(k) e^{-ikct}.$$
Now by the shifting formula,
$$u(x,t) =  F(x - ct).$$
Note the pattern

\begin{diagram}
u(x,t)             && \rTo^{\F}             && \text{DE  for } \uhat(k,t) \\
\dTo              &&                             && \dTo_{\text{solve in t}} \\
u(x,t)             && \lTo                     && \uhat(k,t)
\end{diagram}

\section{Scaling}
Suppose I want to compute
\begin{align*}
\F\{f(bx)\} &= \infint f(bx)e^{-ikx}~dx \qquad z=bx \quad dz = bdx \\
    &=  \infint f(z) e^{-ik\frac{b}{z}}\frac{dz}{b} \\
    &= \frac{1}{b} \fhat \left(\frac{k}{b}\right).
\end{align*}
So
    $$\F\{f(bx)\} = \frac{1}{b} \fhat \left(\frac{h}{b}\right).$$

\section{$\delta$-sequences and the Transform of a $\delta$-function}
A $\delta$-sequence satisfies
$$\infint \delta_{\epsilon}(x)~dx = 1 \quad \text{ and } \lim_{\epsilon \to 0}
\delta_\epsilon(x) = \begin{cases} \infty \quad x = 0 \\ 0 \quad x \neq 0
    \end{cases}.$$
Given any function $f(x)$ that is $f(x) > 0$, $f(x)$ is continuous,
      $$\infint f(x) ~dx = 1,$$
I claim that
$$f_{\epsilon}(x) = \frac{1}{\epsilon}f(\frac{x}{\epsilon})$$
is a $\delta$-sequence.\\~\\
Proof: Note
\begin{align*}
\infint f_{\epsilon}(x)~dx &= \infint f(\frac{x}{\epsilon}) \qquad
    \frac{x}{\epsilon} = z  \qquad \frac{dx}{\epsilon} = dz\\
 &= \infint f(z)~dz \\
 &= 1
\end{align*}
Note this rescaling as $\epsilon \to 0$: (diagram of gaussians getting thinner
and taller.)\\~\\
Consider the Fourier Transform:
$$\F\{f_{\epsilon}(x)\} = \fhat_{\epsilon}(k).$$
First note
$$\fhat_{\epsilon}(0) = \infint f_{\epsilon}(x)e^{-i0x}~dx = 1.$$
Also,
\begin{align*}
\F \{f_{\epsilon}\} &= \F \{ \frac{1}{\epsilon} f\left(\frac{x}{\epsilon}\right)\} \\
    &= \frac{1}{\epsilon} \F\{f\left(\frac{x}{\epsilon}\right)\} \\
    &= \frac{1}{\epsilon} \F\{f\left(\frac{x}{\epsilon}\right)\} \\
    &= \frac{1}{\epsilon} [ \epsilon \fhat(k \epsilon)] \\
    &= \fhat(k \epsilon).
\end{align*}
Now
    $$\lim_{\epsilon \to 0} \fhat_{\epsilon}(k) = \lim_{\epsilon \to 0}
\fhat(k\epsilon)$$
    and for any fixed k,\footnote{Note that we assume the Fourier Transform is
        continuous, which can be proven.}
    $$ \lim_{\epsilon \to 0} \fhat_\epsilon(k) = \fhat(0) = 1.$$
        This is the Fourier Transform
    $$\F\{\delta(x)\} = 1.$$
\end{document}
