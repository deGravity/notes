\documentclass[cm]{article}
\usepackage{macros}
\usepackage{unnumthm}

\newcommand{\uhat}{\hat{u}}

\title{Duhamel's Principle}
\author{Professor Andrew Bernoff}

\begin{document}
\maketitle
{\it The inverse transform of the integral is the integral of the inverse
    transform.} - Duhamel\\~\\

We've talked about arbirtary forcing in the spatial variable--- What about
arbitrary forcing in time?
\\~\\
\ex
$$u_e + u_x = f(t) \delta(x) \qquad - \infty < x < \infty, t > 0$$
$$u(x,0) = 0 \qquad -\infty < x < \infty.$$
Imagine wind blowing with unit speed (diagram on board).
$$u_t + u_x = 0 \longrightarrow u(x,t) = f(x - t).$$
To solve, FT in $x$
\begin{align*}
\Fou \{u(x,t)\} = \uhat(k,t)
\end{align*}
and
\begin{align*}
\Fou \{u_x(x,t)\} &= ik\uhat(k,t) \\
\Fou \{u_t(x,t)\} &= \uhat_t(k,t)
\end{align*}
so
\begin{align*}
\text{DE:}&~ \uhat_t ik\uhat = f(t) \\
\text{IC:}&~ \uhat(k,0) = 0
\end{align*}
Solve with the method of integrating factors:
\begin{align*}
\mu(t) &= e^{\int ik~dt} \\
          &= e^{ikt}
\end{align*}
So
\begin{align*}
e^{ikt}(\uhat_t + ik\uhat) &= f(t)e^{ikt} \\
\frac{d}{dt}(\uhat e^{ikt}) &= f(t) e^{ikt}
\end{align*}
Integrate from $0$ to $t$, because the IC is at $t=0$.
\begin{align*}
\int_0^t \frac{d}{dt} (\uhat e^{ikt})~dt &= \int_0^t f(s) e^{iks}~ds \\
        \uhat e^{ikt} \Big[_0^t &= \int_0^t f(s) e^{iks}~ds \\
        \uhat(k,t)e^{ikt} - \underbrace{\uhat(k,0)}_{=0}1 &= \int_0^t f(s)
        e^{iks}~ds
\end{align*}
and
\begin{align*}
\uhat(k,t) &= e^{-ikt} int_0^t f(s) e^{iks}~ds \\
\uhat(k,t) &= \int_0^t f(s) e^{ik(s-t)}~ds
\end{align*}
\xex
\rec
The method of integrating factors: Suppose
$$y' a(t)y = b(t), y(0) = 0.$$
To solve, multiply by the integrating factor that makes the left hand side an
exact derivative:
$$\mu(t) = e^{\int a(s)~ds}$$
Note
\begin{align*}
    (y(t) e^{\int a(s)~ds})' &= y' e^{\int a(s) ~ds} + y e^{\int a(s)~ds} \\
    &= (y' + a(t) y)e^{\int a(s)~ds}
\end{align*}
So multiplying the DE by $\mu(t)$ yields
\begin{align*}
(y'+a(t)y)\mu(t) &= b(t) \mu(t) \\
(y \mu(t))' &= b(t) \mu(t) \\
\int_0^t (y(s)\mu(s))'~ds &= \int_0^t b(t) \mu(t)~dt \\
y(s) \mu(s) \Big[_0^t &= \int_0^t b(t) \mu(t)~dt \\
y(t) &= \frac{1}{\mu(t)} \int_0^t b(s) \mu(s)~ds
\end{align*}
where
$$\mu(t) = e^{\int_0^s a(s)~ds}.$$
\xrec
\defn
A \emph{t-convolution} is the following
$$f \star g = \int_0^t f(s) (t-s)~ds$$
This comes up often when studying Laplace Transforms.
\xdefn

\begin{align*}
u(x,t) &= \Fou^{-1} \left\{ \int_0^t f(s) e^{ik(s-t)}~ds\right\} \\
          & \int_0^t f(s) \Fou^{-1} e^{ik(s-t)}~ds
\end{align*}
This is \emph{Duhamel's Principle} in action - as long as the integrals converge
absolutely, I can exchange their order.
\begin{align*}
\Fou^{-1}e^{ik(s-t)} &= \delta(x - (s + s)) \\
    &= \delta((x-t)+s)
\end{align*}
(this is likely incorrect) so
\begin{align*}
\int_0^t f(s) \Fou^{-1} e^{ik(s-t)}~ds &= \\
    &= \int_0^t f(s) \delta(x - (s - t))\\
    &= \begin{cases}
    f(t-x) \quad 0 < x < t \\ 
    0 \quad \text{ otherwise}
    \end{cases}
\end{align*}
Graph on board.

\section{Diffusion of a variable point source}
\begin{align*}
\text{DE:}&~ u_t = Du_{xx} + q(t)\delta(x) \\
\text{IC:}&~ u(x,0) = 0
\end{align*}
Diagram on board, something something blowtorch at origin at $t=0$; things are
getting HOT (because blow-torches don't tend to blow cold). As time increases,
this heat distribution grows and spreads.

To solve, FT:
\begin{align*}
\Fou\{u_t\} = \uhat_t \\
\Fou\{u_{xx} &= (ik)^2\uhat = -k^2 \uhat, \Fou\{\delta(x)\} = 1
\end{align*}
\begin{align*}
\text{DE:}&~\uhat_t = -Dk^2\uhat + q(t) \\
\text{IC:}&~ \uhat(k,0) = 0
\end{align*}
So
$$\uhat_t + Dk^2 \uhat = q(t)$$
The EF, $\mu(t) = e^{\int Dk^2~dt} = e^{Dk^2t}$ and
\begin{align*}
e^{Dk^2t}(\uhat_t + Dk^2\uhat) &=e^{Dk^2t}q(t) \\
\frac{d}{dt}\left(\uhat e^{Dk^2t}\right) &= e^{Dk^2t}q(t)
\end{align*}
Integrate w.r.t. $t$ from $0$ to $T$
\begin{align*}
int_0^T \frac{d}{dt} \left( \uhat e^{Dk^2t}\right)~dt &= \int_{t=0}^T e^{Dk^2t}
q(t) \\
\uhat e^{Dk^2t}\Big[_{t=0}^T & = \int_{t=0}^T e^{Dk^2t} q(t) \\
\uhat(k,t) e^{Dk^2 T} - \underbrace{\uhat(k,0)}_{=0}1 &= \int_0^T q(t)
e^{Dk^2t}~dt \\
\uhat(k,t) &= e^{-Dk^2T}\int_0^Tq(t)e^{Dk^2t}~dt\\
\uhat(k,t) &= \int_0^T q(t) e^{-Dk^2(T-t)}~dt
\end{align*}
or
$$\uhat(k,T) = q(T) \star e^{-Dk^2T}.$$
Need to invert the FT
\begin{align*}
u(x,t) &= \Fou^{-1} \{ \int_0^T q(t) e^{-Dk^2(T-t)}~dt\} \\
       &= \int_0^T q(t) \Fou^{-1} \{e^{-Dk^2 (T -t)}~dt\} \\
\end{align*}
Remember
\begin{align*}
\Fou^{-1} \{ e^{-bk^2} \} &= \frac{1}{2 \sqrt{\pi b}} e^{-x^2/4b} \qquad b =
D(T-t) \\
u(x,t) &= \int_0^T q(t) \frac{e^{-x^2/4D(T-t)}}{2\sqrt{\pi D(T-t)}}~dt \\
    &= q(T) \star G(x,T)
\end{align*}
where 
$$G(x,T) = \frac{e^{-x^2/4Dt}}{2\sqrt{\pi D T}}.$$
Superimpose a $G(k,t)$ [corresponding to a $\delta$-function IC at the origin]
with stregth $q(t)$ for every time $0 < t < T$. This formula is complex, but is
simple if we only consider the temperature at $x = 0$.
$$u(0,T) = \frac{1}{2\sqrt{\pi D}} \int_0^T \frac{q(t)}{\sqrt{T - t}}~dt$$
\ex
$$q(t) = 1 \implies u(0,T) = \frac{1}{2\sqrt{\pi D}} \int_0^T \frac{1}{\sqrt{T -
t}}~dt = \frac{T^{1/2}}{\sqrt{\pi D}}.$$
Diagram on board.
\xex
\ex
$$q(t) = t^{\beta}$$
\begin{align*}
U(0,T) &= \frac{1}{2\sqrt{\pi D}} \int_0^T t^{\beta} (T - t)^{-1/2} ~dt \tag{
    $\beta$eta Function} \\
U(0,T) &= \frac{1}{\sqrt{\pi D}} t^{\beta t \frac12} \frac{(\beta !)^2 2^{2
    \beta}}{(2\beta + 1)!}
\end{align*}
\xex
\end{document}
