\documentclass[cm]{article}
\usepackage{macros}

\title{Fields and Waves}
\author{Vatche Sahakian}

\begin{document}

\maketitle

\section{Classical Fields}
\subsection{A continuum limit}

A field theory is described by a set of $N$ fields:
\[ \varphi^a(x^0, x^1, x^2, x^3) = \varphi^a(x) \quad a = 1 \ldots N,\]
where
\[x^0 = ct, x^1 = x, x^2 = y, x^3 = z \]
and $c$ is a characteristic speed of the theory (such and the speed of light or
the speed of sound.) These fields are governed by a Langrangian functional, for
example (and using $ \partial_{\mu} \equiv \frac{\partial}{\partial x^{\mu}} $
for notational convenience)
\[L=\frac{Y}2 \int dx \left( (\partial_0 \varphi)^2 -(\partial_1\varphi)^2\right).\]
\defn[Action, Lagrangian Density]

The action in a field theory is
\[ S = \int dt \  L = \int d^4x \mathcal{L} [\varphi^a, \partial_{\mu} \varphi^a,
  x^{\mu}] \]
where $\mathcal{L}$ is the Lagrangian density, and is defined implicitly by the
above.
\xdefn
The Lagrangian density of the earlier example is
\[ \mathcal{L} = \frac{\sqrt{Y \mu}}{2} \left( (\partial_0 \varphi)^2 -
    (\partial_1 \varphi)^2 \right). \]
Field theories can involve fewer or more dimensions than 4. It is the convention
for this text that time-like dimensions are given the $x^0 = ct$ coordinate. A
commonality among all field theories is that they carry energy that can be
viewed as being stored at every spacetime point.
\subsection{An Action Principle}
The action principle is that the motion undergone by the system will be that
which minimizes the action. Thus we wish to find $\varphi^a$ such that the
variation $\delta S = 0$ (with respect to $\varphi^a$. Following the usual
derivation of Euler's equations we get the conditions
\[ \frac{\partial \mathcal{L}}{\partial \varphi^a} = \partial_{\mu} \left(
  \frac{ \partial \mathcal{L}}{ \partial \partial_{\mu} \varphi^a} \right). \]
This mirrors the case for discrete degrees of freedom:
\[ \frac{\partial \mathcal{L}}{\partial q_k} = \frac{d}{dt} \left( \frac{
  \partial \mathcal{L}}{\partial \dot{q}_k} \right).\]
\subsection{Complex Fields}
A field theory consisting of two decoupled scalar fields can be re-written as
being of one complex field.  For example, for $\varphi^1$ and $\varphi^2$ we can
write $\varphi = \varphi_1 + i \varphi_2$ and use the identities $\varphi_1 =
\frac12 (\varphi + \varphi^*)$ and $\varphi_2 = \frac{1}{2i} ( \varphi -
  \varphi^*)$ to re-write the Lagrangian in terms of $\varphi$ and $\varphi^*$.
If Euler's equations are used treating these as different fields, the resulting
equations of motion will be equivalent.
\section{Symmetries \& Conservation Laws}
\subsection{Noether's Theorem}
Noether's Theorem states that for every continuous symmetry of a physical
system, there exists a conservation law. To understand this, we must first
define what is meant by a continuous symmetry, and what is meant by a
conservation law. First, a continuous system is some deformation of the system,
both of the fields themselves and the coordinates, that preserves the action
regardless of whether or not the field equations of motion are satisfied.
Quantitatively we wrtie this as
\[ \{ \bar{\delta}\varphi^a, \delta x^{\mu} \} \implies \delta S = 0 ,\]
where $\{ \bar{\delta}\varphi^a, \delta x^{\mu} \}$ is the symmetry (technically
just a perturbation if the $\delta S = 0$ condition is not satisfied.)
Propogating the effects of a given perturbation on $S$ to calculate $\delta S$,
and taking care to take into account the dependence of the fields on the
coordinates we find that
\[ \delta S = \int d^4x \left( \frac{\partial \mathcal{L}}{\partial \varphi^a}
    \bar{\delta}\varphi^a + \frac{ \partial \mathcal{L}}{\partial \partial_{\mu}
    \varphi^a} \partial_{\mu} \bar{\delta}\varphi^a + \partial_{\mu} (\delta
      x^{\mu} \mathcal{L}) \right).\]

Now we assume that this pertubation is a symmetry (so set $\delta S = 0$ and
that the equations of motion are satisfied. Then we get that
\[ \delta S = 0 = \int d^4x \partial_{\mu} j^{\mu} \]
where
\[ j^{\mu} = \frac{\partial \mathcal{L}}{\partial \partial_{\mu} \varphi^a}
\bar{\delta}\varphi^a + \delta x^{\mu} \mathcal{L} \]
is the conserved current associated with the symmetry. To see this, note that
\[ \partial_\mu j^\mu = 0 \ \partial_0 j^0 + \partial_i j^i \]
so
\[ \int d^3x \partial_0 j^0 + \int d^3 x \vec{\nabla} \cdot \vec{j} = 0.\]
Using the divergence theorem and defining the Noether Change $Q$ to be
\[Q = \int d^3x \frac{j^0}{c}\]
we can re-arrange this equation to see
\[ \frac{dQ}{dt} = - \oint d \vec{A} \cdot \vec{j}.\]
Thus $\frac{j^0}{c}$ is the density of the Noether Charge which is conserved
over regions in space with the $j^i$ components describing the flux out of the
regions.
\end{document}
