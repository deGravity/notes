\documentclass[cm]{article}
\usepackage{macros}
\usepackage{wasysym}
\newcommand{\ghat}{\hat{g}}
\newcommand{\fhat}{\hat{f}}
\newcommand{\yhat}{\hat{y}}
\renewcommand{\F}{\mathscr F}
\newcommand{\uhat}{\hat{u}}
\newcommand{\What}{\hat{W}}
\newcommand{\hhat}{\hat{h}}
\renewcommand{\infint}{\int_{-\infty}^{\infty}}


\begin{document}
\section{Homework Issues}
Turn it in tomorrow. A head start:
\begin{align*}
g'' - 2 g' + 2g &= \delta(x) \\
\F \{ g'' - 2 g' + 2g \} &= \F \{ \delta(x) \} \\
(-k^2 - 2 ik + 2) \ghat &= 1\\
\ghat &= \frac{1}{-k^2 - 2 ik + 2}
\end{align*}
Note that
\begin{align*}
(ik)^2 - 2ik + 2 &= (ik - i)^2 + 1\\
&= [(ik-1+i)(ik-1-i)]
\end{align*}
so
\begin{align*}
\ghat(k) &= \frac{1}{(ik-1+i)(ik-1-i)} \\
&= \frac{A}{ik - 1 + i} + \frac{B}{ik - 1 - i} \\
&= \frac{-\frac{1}{2i}}{(ik-1+i} + \frac{\frac{1}{2i}}{(ik - 1 -i)} \\
&= \frac{1}{2i} \left[ \frac{1}{(ik - 1 - i} - \frac{1}{ik - 1 + i}\right]
\end{align*}
We then need to figure out the inverse transform of this:
\begin{align*}
\F \{ H(x) e^{-ax} \} &= \int_0^{\infty} e^(-a-ik)x \\
&= \frac{e^{(-a-ik)x}}{-a-ik}\Big|_{x=0}^{\infty} \\
&= \frac{1}{a+ik} \qquad \Re\{a\} > 0
\end{align*}
Suppose $a = \alpha + i \beta$. Then
$$ = \frac{e^{-\alpha - i \beta - i k x}}{-a-ik} = e^{- \alpha x} [ \ldots$$

\section{Fun Fact}
$$I = \infint e^{-a x^2 /2}~dx = \sqrt{\frac{2 \pi}{a}}.$$
Proof:

\begin{align*}
I^2 &= \infint e^{-ax^2/2}~dx \infint e^{-ay^2/2}~dy\\
&= \infint \infint e^{-\frac{a}{2} (x^2 + y^2)}~dx~dy \\
&= \int_{\theta = 0}^{2\pi} \int_{r = 0}^{\infty} e^{-a \frac{r^2}{2}} ~r~dr~d\theta \\
&= \frac{2 \pi}{a} \int_0^{\infty} e^{\frac{-ar^2}{2}} a~r~dr \\
&= \frac{2 \pi}{a} ( - e^{-a r^2 / 2}) \Big|_{r=0}^{\infty} \\
I^2 &= \frac{2 \pi}{a} \\
I &= \sqrt{\frac{2 \pi}{a}}
\end{align*}

\section{Cauchy Problem for the Heat Equation}
\begin{align*}
\text{DE:}&~u_t = Du_xx \qquad - \infty < x < \infty, t > 0 \\
\text{IC:}&~ u(x,0) = f(x)\qquad - \infty < x < \infty \\
\text{BC:}&~ \max |u(x,t)| \text{ is bounded for all } t > 0
\end{align*}
Also, assume $\max|f(x)|$ is bounded. Do we always need this boundary condition? Yes, for a physical solution. How do we solve this thing? (Picture 1 in Notebook). Solution: Use the Fourier Transform.
\begin{align*}
\F \{ u(x,t) \} &= \uhat(k,t) \\
\F \{ u_t(x,t) \} &= \uhat_t(k,t) \\
\F \{ u_xx(x,t) \} &= (ik)^2 \uhat(k,t) \\
&= -k^2 \uhat(k,t),
\end{align*}
So
$$u_t = Du_xx \implies \uhat_t = -Dk^2 \uhat,$$
\begin{equation*}
\uhat(k,t) = A(k) e^{-Dk^2 t} \tag{*}
\end{equation*}
From the IC
$$\F \{u(x,0) \} = \uhat(k,0) = \F \{ f(x) \} = \fhat(k),$$
but from *,
$$\uhat(k,0) = A(k) = \fhat(k) \implies \uhat(k,t) = \fhat(k) e^{-Dk^2 t}.$$
We need to known
$$\F \{e^{(-(Dt)k^2}\}.$$
\section{Fourier Transform of a Gaussian}
\begin{align*}
\F \{ e^{-ax^2/2} \} = \infint e^{-ax^2/2 + ikx}~dx
\end{align*}
Let's complete the square
\begin{align*}
\frac{ax^2}{2} + ikx &= \frac{a}{2} [x^2 + \frac{2i k}{a} x] \\
&= \frac{a}{2} [ (x + \frac{ik}{a})^2 - (\frac{ik}{a})^2] \\
&= \frac{a}{2} (x + \frac{ik}{a})^2 + \frac{k^2}{2a})
\end{align*}
So
\begin{align*}
\F \{ e^{-ax^2/2}\} &= \infint e^{-  \frac{a}{2} (x + \frac{ik}{a})^2 + \frac{k^2}{2a})}~dx \\
&= e^{-\frac{k^2}{2a}} \int_{x = - \infty}^{\infty} e^{- \frac{a}{2} (x + \frac{ik}{a})^2}~dx
\end{align*}
Let 
\begin{align*}
z &= x + \frac{ik}{a} \\
dz &= dx.
\end{align*}
What is $\infty + \frac{ik}{a}$? Answer: $\infty$. See picture 2 in notebook. This function is analytic everywhere. So
\begin{align*}
\F \{ e^{-ax^2/2} \}&\stackrel{\footnotemark}{=} e^{-\frac{k^2}{2a}} \infint e^{- \frac{a}{2} z^2} ~dz \\
&= e^{- \frac{k^2}{2a}} \sqrt{\frac{ 2 \pi}{a}}
\end{align*}
So
\begin{equation*}
\F \{ e^{a x^2/2} \} = \sqrt{ \frac{2 \pi}{a}} e^{-k^2/2a}. \tag{\smiley}
\end{equation*}
The Fourier Transform of a Gaussian is a Gaussian! \footnotetext{with a little help from Math 136.}
Also
$$\F \{ \frac{1}{2\sqrt{\pi b}} e^{-x^2/4b} \} = e^{-bk^2}.$$
In \smiley, set
$$b = \frac{1}{2a} \implies a = \frac{1}{2b}$$
and multiply by
$$ \sqrt{\frac{a}{2\pi}} = \frac{1}{2 \sqrt{\pi b}}.$$
Finally, we have $(b = Dt)$
\begin{align*}
\F^{-1} \{ e^{-Dk^2 t}\} &= G(x,t) \\ 
&= \frac{1}{2 \sqrt{\pi D t}} e^{-x^2 / 4 Dt}.
\end{align*}
This is the Green's function (or the ``Kernel'') of the heat equation.
\subsection{Solutions of the Cauchy problems for...}
\enum
\item[i)] $f(x) = \delta(x)$ delta
\item[ii)] $f(x) = f(x)$ arbitrary
\item[iii)] $f(x) = H(x)$ Heaviside
\xenum
~\\
\enum
\item[i)] If $f(x) = \delta(x)$, $\F \{ \delta (x) \} = 1$. Then
$$ u(x,t) = G(x,t)  \frac{e^{-x^2/4Dt}}{2 \sqrt{\pi D t}}.$$
See picture 3 in notebook. Width scales like $2 \sqrt{Dt}$ spreading. Height scales like $\frac{1}{2 \sqrt{\pi D t}}$ decreasing. Area $ = 1$. Self similar diffusion - it spreads out, but maintains its characteristic shape.
\item[ii)] $f(x) = f(x)$
\begin{align*}
\uhat(k,t) &= \fhat(k,t) e^{-Dtk^2} \\
&\implies \\
u(x,t) &= f(x) \star G(k,t) = \frac{1}{2\sqrt{\pi D t}} \infint f(x-y) e^{-y^2 / 4 D t} ~dy
\end{align*}
Poisson Integral Formula for the solution to the Cauchy problem. It is a continuous superposition of the solutions to the problem. It is a sum of many delta functions.
\item[iii)] $f(x) = H(x)$
$$u(x,t) = \frac{1}{2 \sqrt{\pi D t}} \infint H(x-y) e^{-y^2 / 4 Dt} ~dy.$$
$$H(x-y) = \begin{cases} 1 \quad x > y \\ 0 \quad x < y \end{cases}.$$
So
$$u(x,t) = \frac{1}{2 \sqrt{\pi D t}} \int_{-\infty}^{x} e^{-y^2 / 4 Dt} ~dy.$$
Let $z = \frac{y}{2 \sqrt{D t}}$ then $dz = \frac{dy}{2 \sqrt{Dt}}$.
So
$$ u(x,t) = \frac{1}{\sqrt{\pi}} \int_{-\infty}^{\frac{x}{2\sqrt{Dt}}} e^{-z^2}~dz.$$
Remember the error function
$$\erf(w) = \frac{2}{\sqrt{\pi}} \int_0^w e^{-s^2}~ds.$$
See picture 4 in notebook.
\begin{align*}
u(x,t) &= \frac{1}{2} [ \frac{2}{\sqrt{\pi}} \int_{-\infty}^0 e^{-w^2}~dw + \int_0^{\frac{x}{2 \sqrt{D t}}} e^{-w^2} ~dw \\
&= \frac{1}{2} [ - \erf(- \infty) + \erf( \frac{x}{2\sqrt{Dt}})] \\
&= \frac{1}{2} [ 1 + \erf(\frac{x}{2 \sqrt{Dt}})]
\end{align*}
See picture 5 in notebook.
\xenum
\end{document}