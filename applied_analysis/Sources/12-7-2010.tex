\documentclass{article}
\usepackage{macros}
\usepackage{numthm}
\usepackage{ esint }

\newcommand{\x}{\vec{x}}
\newcommand{\y}{\vec{y}}
\renewcommand{\u}{\vec{u}}
\newcommand{\uhat}{\hat{u}}
\newcommand{\nhat}{\hat{n}}
\newcommand{\q}{\vec{q}}
\newcommand{\ihat}{\hat{i}}
\newcommand{\jhat}{\hat{j}}
\renewcommand{\t}{\vec{t}}

\title{So Long and Thanks for all the Fish}
\author{Professor Bernoff}
\date{12/7/2010}

\begin{document}
\maketitle
\subsection{Divergence Theorem}
In $\R^n$, given a vector $\u = u_1 \ihat + u_2 \jhat$ its divergence is $\nabla
\u = \frac{\partial u_1}{\partial x} + \frac{\partial u_2}{\partial y}$ and
$\int\int_a \nabla \cdot u~dV = \oiint \u \cdot \nhat ~ds$, where $\nhat$ is an
outwardly pointed normal.

\subsection{Heat Equation in $\R^n$}
We will use $\R^2$, but the derivation generalizes. Let $u(\x, \t)$ be the
temperature in $\Omega$. The heat energy is given by
\[ Q = \iiint_{\Omega} c_p \rho u~dV\]
where $c_p$ is the specific heat per unit mass $ =
\frac{\text{Energy}}{\text{degree mass}}$ and $\rho = $ density $ =
\frac{\text{mass}}{\text{volume}}$.

Assume $c_p$ and $\rho$ are constant. Then
\[ \frac{dQ}{dt} = c_p \rho \iiint_{\Omega} \frac{\partial u}{\partial t}~dV\]
Fourier's Law of cooling says that the heat flux is proportional to the
temperature gradient
\[ \q = -k \nabla u\]
where $k = $ thermal conductivity $ = \frac{\text{energy}}{\text{degree
length}}.$

So
\begin{align*}
\frac{dQ}{dt} &= - \{ \text{flux out of $\Omega$ of heat} \} \\
        &= - \oiint_{\partial \Omega} \q \cdot \nhat ~ds \\
        &\stackrel{\text{div}}{=} - \iiint_{\Omega} \nabla \cdot q~dV
\end{align*}
But the volume $\Omega$ is arbitrary, s the two integrands must be equal.
\[c_p \rho = - \nabla \cdot \q\]
but
\[ \q = - k \nabla u.\]
So
\[c_p \rho \frac{\partial u}{\partial t} = + k \nabla \cdot ( \nabla u)\]
But
\[ \nabla \cdot \nabla = \nabla^2 u\]
so
\[\frac{\partial u}{\partial \tau} = D \nabla^2 u\]
\[D = \frac{k}{c_p\rho}.\]

\subsection{Steady States}
Suppose we have the Dirichlet problem for the heat equation.
\begin{align*}
\text{DE:}&~ u_t = D\nabla^2 u \qquad \text{in } \Omega, t > 0 \\
\text{BC:}&~ u |_{\partial \Omega} = f(\x) \qquad \text{ on } \partial \Omega, t
> 0 \\
\text{IC:}&~ u(\x, 0) = g(\x) \qquad \text{in } \Omega
\end{align*}
The solution approaches a steady state, that is where $u_t = 0$. Call this state
$\phi(\x)$. $\phi$ satisfies Laplace's Equation
\begin{align*}
\nabla^2 \phi &= 0 \qquad \text{ in } \Omega \\
\phi &= f(\x) \qquad \text{ on } \partial \Omega.
\end{align*}
This solution exists (hard!) and is unique (easier).
\subsubsection{Proof of Uniqueness}
Suppose we have two solutions, $\phi_1$ and $\phi_2$. Consider $\Psi = \phi_1 =
\phi_2$. $\Psi$ satisfies a homogereous Laplace Equation
\begin{align*}
\nabla^2 \Psi &= 0 \qquad \text{ in } \Omega \\
\Psi &= 0 \qquad \text{ on } \partial \Omega
\end{align*}
Let's prove this with an energy method, but first, a vector identity.
\begin{align*}
    \nabla \cdot (\Psi \nabla \Psi) &= \Psi \nabla^2 \Psi + \nabla \Psi \cdot
    \nabla \Psi \\
        &= \Psi \nabla^2 \Psi + | \nabla \Psi |^2
\end{align*}
Also, if $\Psi$ is harmonic (i.e. satisfies $\nabla^2 \Psi = 0$) then
\[ \nabla \cdot ( \Psi \nabla \Psi) = | \nabla \Psi |^2.\]

Now consider
\begin{align*}
\iiint_\Omega |\nabla \Psi|^2 ~dV &= \iiint_\Omega \nabla \cdot (\Psi \nabla \Psi)~dV \\
        &\stackrel{\text{div}}{=} \oiint_{\partial \Omega} \Psi \nabla \Psi
        \cdot \nhat~dS
\end{align*}
But $\Psi = 0$ on $\partial \Omega$, so
\[ \iiint_{\Omega} |\nabla \Psi|^2 ~dV = 0\]
which implies $|\nabla \Psi|^2 = 0 \implies \nabla \Psi = 0 \implies \Psi = $
constant, but $\Psi = 0$ on $\partial \Omega \implies \Psi = 0$ identically.
Therefore $\phi_1 = \phi_2$, so solutions to Laplace's Equation are unique.

\subsection{Heat Equation in a Square}
This was a problem on a previous final exam:

Solve
\begin{align*}
\text{DE:}&~ u_t = \nabla^2 u \qquad 2 < x < \pi, 0 < y < \pi, t > 0 \\
\text{IC:}&~ u(x,y,0) = f(x,y) \\
\text{BC:}&~ u(x,0,t) = u(x,\pi,t) = 0 \qquad 2 < x < \pi, t > 0 \\
&~ u(0,y,t) = u(\pi,y,t) = 0 \qquad 0 < y < \pi, t > 0
\end{align*}

The boundary equations asy that $u = 0$ on $\partial \square.$

\subsubsection{Solution}
Use separation of variables. Let
\[ u(x,y,t) = T(t) \Psi(x,y)\]
\begin{align*}
\text{DE:}&~ u_t = \nabla^2u \implies T_t \Psi = T \nabla^2 \Psi \\
        &~ \frac{T_t}{T} = \frac{\nabla^2 \Psi}{\Psi} = - \lambda
\end{align*}
The $T$-equation
\[ T_t + \lambda T = 0 \implies T(t) = e^{-\lambda t}.\]
The $\Psi$-equation we have actually seen before. From the BC on $\partial
\square$, $u(x,y,t) = \Psi(x,y) T(t) = 0$, so $\Psi$ vanishes on the boundary
also.
\begin{align*}
\text{DE:}&~ \nabla^2 \Psi + \lambda \Psi = 0 \qquad \text{ in } \Omega\\
\text{BC:}&~ \Psi(x,y) = 0 \quad \text{ on } \partial \Omega
\end{align*}
I will show that $\lambda$ is real and positive - just assume it for the moment.
Separate $\Psi(x,y) = X(x)Y(y)$.
\begin{align*}
    \nabla^2 \Psi = \Psi_{xx} + \Psi_{yy} = X_{xx} Y + XY_yy + \lambda XY = 0
\end{align*}
Divide by $X,Y$
\[\underbrace{\frac{X_xx}{X}}_{= - \mu_1} + \underbrace{\frac{Y_yy}{Y}}_{= -
    \mu_2} = - \lambda\]
$X$-equation
\[X_xx + \mu_1 X = 0\]
BC'- $\implies$ $X(0) = X(\pi)$\\
$Y$-equation
\[Y_yy + \mu_2 Y = 0 \qquad Y(0) = Y(\pi) = 0 \]
So
\[X(x) = X_n(x) = \sin(nx) \qquad n = 1, 2, 3, \ldots\]
$\mu_1 = n^2$ and
\[Y(y) = Y_m(y) = \sin(my) \qquad m = 1, 2, 3, \ldots\]
$\mu_2 = m^2$. So the solution for $\Psi$ is
\[\Psi_{mn} = X_n(x)Y_m(x) = \sin(nx)\sim(mx) \]
$\lambda_{mn} = mu_1 + \mu_2 = n^2 + m^2$. So these are the eigenfuctions and
eigenvalues of the Helmholtz equation.

Note
\[u(x,y,t) = \sum_{n = 1}^{\infty} \sum_{m = 1}^{\infty} c_{nm} \Psi_{nm}(x,y)
    e^{-\lambda_{nm}t}\]
We need to determine the $c_{nm}$s. What about the IC?
\begin{align*}
u(x,y,0) &= \sum_{n = 1}^{\infty} \sum_{m = 1}^{\infty} c_{nm} \Psi_{nm}(x,y) \\
         &= f(x,y)
\end{align*}
I need an orthogonality condition.
\subsubsection{Orthogonality}
\begin{align*}
\langle \langle \Psi_{nm}, \Psi_{pq} \rangle \rangle &= \int_{y=0}^{\pi} \int_{x
    = 0}^\pi \Psi_{nm} \Psi_{pq} ~dx~dy \\
    &= \begin{cases} 0 \qquad n \neq p \text{ or } m \neq q \\
    \frac{\pi^2}{4} \qquad n = p \text{ and } m = q . \end{cases}
\end{align*}
Cheap proof:
\begin{align*}
\langle \langle \Psi_{nm}, \Psi_{pq} \rangle \rangle &= \int_{y = 0}^\pi \int_{x
    = 0}^\pi \sin(nx) \sin(my) \sin(px) \sin(qy) ~dx~dy \\
    &= \underbrace{ \int_{y = 0}^\pi \sin(my) \sin(qy)~dy}_{= 0 m \quad m \neq
        q} \underbrace{\int_{x = 0}^\pi \sin(nx) \sin(px)~dx}_{=0 \quad n \neq
            p}
\end{align*}
So
\[f(x,y) = \sum_{n = 1}^\infty \sum_{m = 1}^\infty c_{nm} \Psi_{nm}(x,y) \]
and
\[\langle \langle \Psi_{pq}, f(x,y) \rangle \rangle = c_{pq} \frac{\pi^2}{4}\]
So
\[c_{pq} = \frac{4}{\pi^2} \int_0^\pi \int_0^\pi f(x,y) \sin(px)
    \sin(qy)~dx~dy\]
and
\[ u(x,y,t) = \sum_{n = 1}^\infty \sum_{m = 1}^\infty e^{-(n^2 + m^2) t} c_{nm}
\sin(nx) \sin(mx)\]

\thm
Suppose $\Psi(x,y)$ is a non-zero solution of $\ast$ for some eigenvalue
$\lambda$. Then $\lambda$ is real and positive.
\xthm
\prf
Consider
\begin{align*}
\iint_{\Omega} \nabla \cdot (\Psi \nabla \Psi)~dV &\stackrel{\text{div}}{=} 
\oint_{\partial \Omega} \Psi \nabla \Psi \cdot \nhat~ds \\
    &= 0 \qquad \text{because $\Psi |_{\partial \Omega} = 0$; but} \\
    &= \iint_{Omega} \Psi \nabla^2 \Psi + |\nabla \Psi|^2~dV \qquad \text{But
        $\nabla^2 \Psi = - \lambda \Psi$} \\
    &= - \lambda \iint \Psi^2~dV + \iint|\nabla \Psi|^2~dV \\
    &= 0,
\end{align*}
so
\[ \lambda = \frac{\iint |\nabla \Psi|^2~dX}{\iint |\Psi|^2~dV}.\]
Thus $\lambda$ is positive. To show it is real:

\begin{align*}
\iint_{\Omega} \nabla \cdot (\Psi^{\ast} \nabla \Psi)~dV
&\stackrel{\text{div}}{=} \oint_{\partial \Omega} \Psi^{\ast} \nabla \Psi \cdot
\nhat~ds \\ &= 0 \qquad \text{because $\Psi |_{\partial \Omega} = 0$; but} \\&=
\iint_{Omega} \Psi \nabla^2 \Psi + |\nabla \Psi|^2~dV \qquad \text{But $\nabla^2
    \Psi = - \lambda \Psi$} \\ &= - \lambda \iint \Psi^2~dV + \iint|\nabla
    \Psi|^2~dV \\ &= 0.
\end{align*}    
\xprf
\subsubsection{``Expensive'' Proof of Orthogonality}
Let $\Psi$, $\phi$ satisfy $\nabla^2 \Psi + \lambda \Psi = 0$, $\nabla^2 \phi
    + \mu \phi = 0$. We wish to show
\[ \iint_\Omega \Psi \phi~dV = 0 \qquad \text{ if } \mu \neq \lambda\]
\prf
Consider
\begin{align*}
\iint \nabla \cdot [\phi \nabla \Psi - \Psi \nabla \phi]~dV = \oint (\phi
        \nabla \Psi - \Psi \nabla \phi) \cdot \nhat~ds
    = 0
\end{align*}
since $\pi = \Psi = 0$ on $\partial \Omega$. But
\begin{align*}
0 &= \iint \nabla \cdot (\phi \nabla \Psi) - \nabla \cdot ( \Psi \nabla \phi)~dV
    \\
  &= \iint \phi \nabla^2 \Psi + \nabla \phi \cdot \nabla \Psi - \Psi \nabla^2
  \phi - \nabla \phi \cdot \nabla \Psi~dV \\
  &= \iint \phi \nabla^2 \Psi - \Psi \nabla^2 \phi ~dV \qquad \text{ But
      $\nabla^2 \Phi = - \lambda \Psi$, $\nabla^2 \phi = - \mu \phi$} \\
  &= - \iint \lambda \phi \Psi - \mu \Psi \phi~dV \\
  &= (\mu + \lambda) \iint \Psi \phi ~dV \\
  &= 0
\end{align*}
which implies either $\lambda = m$ or $\iint_{\Omega} \Psi \phi~dV / 0$.
\xprf
\end{document}
