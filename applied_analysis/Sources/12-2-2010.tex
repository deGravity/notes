\documentclass[cm]{article}
\usepackage{macros}
\usepackage{unnumthm}

\newcommand{\Rtilde}{\tilde{R}}

\title{Vibrations of a Drum}
\author{Professor Bernoff}
\date{12/2/2010}

\begin{document}
\maketitle
The oscillations of a drum are governed by the wave equation;
\begin{align*}
    \text{DE:}&~ u_{tt} = c^2 \nabla^2 \qquad \text{in } \Omega\\
    \text{BC:}&~ u= 0 \qquad \text{on } \partial \Omega, t > 0 \\
    \text{IC:}&~ u_t(x,y,0) = g(x,y).
\end{align*}
Here $u(x,y,t)$ is the displacement of the membrane.

First, let's separate the $t$ variable
    $$u(x,y,t) = T(t) \Psi(x,y).$$
The DE tells
    $$T_tt \Psi = c^2 \nabla^2 \Psi T.$$
So
   % $$\underbrace{\frac{T_{tt}{c^2T}}_{\text{fn of t}} =
   % \underbrace{\frac{\nabla^2 \Psi}{\Psi}}_{\text{fn of x,y}} = -\lambda.$$
We claim that $\lambda$ is real and positive. We now solve the $T$ equation;
    $$T_{tt} + c^2 \lambda T = 0.$$
So
    $$T(t) = A \cos(\omega t) + B \sin(\omega t)$$
    $$ \omega = c \sqrt{\lambda}$$
And now the $\Psi$-equation. Since
    $$u(x,y,t) = T(t) \Psi(x,y) = 0 \qquad \text{on } \partial \Omega,$$
for a non-trivial solution
    $$T(t) \neq 0 \implies \Psi(x,y) = 0 \qquad \text{ on } \partial \omega.$$
This yields the {\bf Helmholtz Problem}
    \begin{align*}
    \nabla^2 \Psi + \lambda \Psi = 0 \qquad \text{ in } \Omega \\
    \Psi = 0 \qquad \text{ on } \partial \Omega
    \end{align*}
This problem has a countable set of real positive eigenvalues for $\Omega$ being
a simply connected compact domain.\footnote{The extent to which these
restrictions can be relaxed is an open question in analysis.} Note $\lambda_n =
\frac{\omega_n^2}{c^2}$ for $n = 1, 2, 3, \ldots$ yields the oscillation
frequencies of the drum.

Question: Does knowing $\{\lambda_n\}$ tell you the shape of the drum? This lead
to a very famous paper from Mark Kac; ``Can you hear the shape of a drum?''

Time to play the Bongos. Consider $\Omega$ to be a disc of radius $a$ centered
at the origin, and parameterize it in polar form by $r$ and $\theta$. Let's find
the eigenvalues of the Helmholtz Equation.

Let $\Psi = \Psi(r,\theta)$.
\begin{align*}
    \text{DE:}&~ \nabla^2\Psi + \lambda\Psi = \Psi_{rr} + \frac{1}{r} \Psi_r +
        \frac{1}{r^2} \Psi_{\theta \theta} + \lambda \Psi = 0 \\
    \text{DC:}&~ \Psi(a, \theta) = 0.
\end{align*}
We proceed by separation of variables.
    $$\Psi(r,\theta) = \R(n) \Theta(\theta)$$
so
    $$R_{rr} \Theta + \frac{1}{r} R_r \Theta + \frac{1}{r^2} R \Theta_{\theta
        \theta} + \lambda R \Theta = 0.$$
Divide by $\frac{\R \Theta}{r^2}.$
$$\frac{R_{rr} + \frac{1}{r} R_r + \lambda R}{\frac{R}{r^2}} = -
\frac{\Theta_{\theta \theta}}{\Theta} = \mu.$$
First we solve the $\Theta$-equation.
$$ \Theta_{\theta \theta} + \mu \Theta = 0 \qquad 0 \leq \Theta \leq 2 \pi$$
I want $\Theta$ to be $2\pi$-periodic.
Claim
\begin{align*}
\Theta_0 &= 1, \quad \mu_0 = 0 \\
\Theta_n &= D_n \cos(n \theta) + E_n \sin( n \theta), \quad \mu = n^2
\end{align*}
are solutions. Now we solve the $R$-equation. For $\mu_n = n^2$, $n = 0,1,2,
\ldots$, I see
    $$R_{rr} + \frac{1}{r} R_r + (\lambda - \frac{n^2}{r^2})R = 0.$$
Recall that this is Bessel's Equation of order $n$.\footnote{Note that this is
why we normally have integer orders. If we had a wedge, we would have
fractional orders.} Also, I want $R(0)$ bounded and
$$\Psi(a, \theta) = R(a) \Theta(\theta) = 0 \implies R(a) = 0.$$
We can scale out $\lambda$. Let $z = \sqrt{\lambda} r$, then $R = \Rtilde$
$$ \frac{d}{dz} = \frac{dr}{dz} \frac{d}{dr} = \frac{1}{\lambda} \frac{d}{dr}
\iff \sqrt{\lambda} \frac{d}{dz} = \frac{d}{dr}.$$
So
$$\lambda \Rtilde_{zz} + \frac{\lambda}{z}\Rtilde_z + \left( \lambda -
        \frac{\lambda n^2}{z^2}\right) \Rtilde = 0.$$
Divide by $\lambda$ to obtain
$$\Rtilde_{zz} + \frac{1}{z} \Rtilde_z + \left(1 - \frac{n^2}{z^2}\right)\Rtilde
= 0.$$
This is a Bessel's Equation of order $n$.
$$\Rtilde(z) = \beta J_n(z) + \gamma \mathbb Y_n(z).$$
Note that $\lim_{z \to 0} \mathbb Y_n(z) = -\infty \implies$ set
$\gamma = 0$ (and $\beta = 1$).
So
$$\tilde{R}(z) = J_n(z)$$
and
$$R(r) = J_n(\sqrt{\lambda}r).$$
Applying the BC at $r = a$
$$R(a) = J_n(\sqrt{\lambda} a) = 0 \implies \sqrt{\lambda} a = \alpha_{np}.$$
Where $\alpha_{np}$ is the $p$th positive zero of $J_n$. Example diagram on
board. Let
$$\lambda_{np} = \left( \frac{\alpha_{np}}{a}\right)^2.$$
So the eigenfunctions and eigenvalues of $\Psi$ are;
\begin{align*}
n = 0:~ \Psi_{0p} &= \Theta_0 R = J_0(\alpha_{op} \frac{r}{a}) \\
\lambda_{0p} &= \left( \frac{\alpha_{0p}}{a}\right)^2\\
n = 1,2,3, \ldots :~ \Psi_{np}^c &= \cos(n \theta) J_n ( \alpha_{np}
        \frac{r}{a})\\
\Psi_{np}^s &= \sin(n \theta) J_n ( \alpha_{np} \frac{r}{a}) \\
\lambda_{np} &= \left(\frac{\alpha_{np}}{a}\right)^2.
\end{align*}
So the oscillation modes are
\begin{align*}
u(r,\theta, t) ~=&~ \cos( \omega_{np}t) \Psi_{np}^c \\
               &~ \sin( \omega_{np}t) \Psi_{np}^c \\
               &~ \cos( \omega_{np}t) \Psi_{np}^s \\
               &~ \sin( \omega_{np}t) \Psi_{np}^c \\
               &~ \sin( \omega_{0p}t) \Psi_{0p} \\
               &~ \cos( \omega_{0p}t) \Psi_{0p}
\end{align*}
where $\omega_{np} = c \sqrt{\lambda_{np}}$ and $n = 1, 2, 3, \ldots$.

Suppose I wish to love
\begin{align*}
\text{DE:}&~ u_{tt} = c^2 \nabla^2u \qquad r < a \\
\text{BC:}&~ u(a,\theta,t) = 0 \\
\text{IC:}&~ u(r,\theta,0) = f(r,\theta) \\
&~ u_t(r,\theta,0) = 0.
\end{align*}
The solution must be expressed in terms of these modes
$$u(r,\theta,t) = \sum_{p=1}^{\infty} A_{0p} \underbrace{J_0 (\alpha_{0p}
        \frac{r}{a})}_{\Psi_{0p}}
\cos(\omega_{0p} t) + \sum_{n = 1}^{\infty} \sum_{p = 1}^{\infty} [ A_{np}
\underbrace{J_n( \alpha_{np} \frac{r}{a})  \cos(n \theta)  }_{\Psi_{np}^c}+
    B_{np} \underbrace{J_n(\alpha_{np}
\frac{r}{a}) \sin(n \theta)}_{\Psi_{np}^s} ]\cos(\omega_{np}t).$$
Remember
$$\{1, \cos(n \theta), \sin(n \theta) \}$$
are orthogonal for
$$\langle h,g\rangle = \int_0^{2\pi} hg ~d\theta.$$
Also, the set
$$\{J_n(\alpha_{np} \frac{r}{a}\}$$
are orthogonal for the inner-product
$$[h,g] = \int_0^a hg r~dr.$$
So for $\{\Psi_{0p}, \Psi_{np}^c, \Psi_{np}^s\}$ the functions are orthogonal
\begin{align*}
\langle \langle \Psi_1, \Psi_2 \rangle \rangle &= \int_0^{2\pi} \int_0^a \Psi_1
\Psi_2 r~dr~d\theta \\
&= \int_{\Omega} \Psi_1 \Psi_2 ~dA.
\end{align*}
So
\begin{align*}
A_{0p} &= \frac{\langle \langle \Psi_{0p}, f(r,\theta) \rangle
\rangle}{\langle \langle \Psi_{0p}, \Psi_{0p} \rangle \rangle} \\
A_{np} &= \frac{\langle \langle \Psi_{np}^c, f(r,\theta) \rangle
\rangle}{\langle \langle \Psi_{np}, \Psi_{np} \rangle \rangle} \\
B_{np} &=  \ldots
\end{align*}
\end{document}
